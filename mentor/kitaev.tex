\documentclass[a4paper, fontsize = 14pt]{article}
\usepackage{hyperref}
\usepackage[warn]{mathtext}
\usepackage[english, russian
]{babel}
\usepackage[utf8x]{inputenc} 
 
%математика
\usepackage[mathscr]{eucal}
\usepackage{amsmath,amsfonts,amssymb,amsthm,mathtools}
\usepackage{icomma}
\usepackage{wasysym}
\usepackage{mathrsfs}
\usepackage[italicdiff]{physics}
 
%оформление текста
\usepackage{setspace}
\onehalfspacing
\usepackage{indentfirst}
\usepackage{scrextend}
 
%геометрия
\usepackage{geometry}
\geometry{left=25mm,right=25mm,
 top=25mm,bottom=30mm}
 
%графика
\usepackage{wrapfig}
\usepackage{graphicx}
\usepackage{pgfplots}
\usepackage{tikz}
\RequirePackage{caption}
\DeclareCaptionLabelSeparator{defffis}{ --- }
\captionsetup{justification=centering,labelsep=defffis}
 
%таблицы
\usepackage{array,tabularx,tabulary,booktabs} 
\usepackage{longtable}  
\usepackage{multirow} 
 
%ссылки
\usepackage{hyperref}
\usepackage{xcolor}
\definecolor{grn}{HTML}{57A14F} %зеленый
\definecolor{rd}{HTML}{E53C44} %красный 
\definecolor{bl}{HTML}{282691} %синий 
\definecolor{bbl}{HTML}{001B6C} %темно-синий
\hypersetup{		
    colorlinks=true,       	
    linkcolor=bbl,          % внутренние ссылки
    citecolor=rd,          % на библиографию
    filecolor=magenta,      % на файлы
    urlcolor=bl           %внешние источники
}
 
% Колонтитулы
\usepackage{fancyhdr} 
 	\pagestyle{fancy}
 	\renewcommand{\headrulewidth}{0.15mm}  
 	\renewcommand{\footrulewidth}{0.15mm}
 	\lfoot{}
 	\rfoot{\thepage}
 	\cfoot{}
 	\rhead{}
 	\chead{}
 	\lhead{}
 
 
\begin{document}

\begin{center} \textbf{Kitaev Algorithm}
\end{center} 

The magnetic moment of our artificial atom is:

\begin{equation*}
	\mu = S \hbar |d \omega_{01}/d\Phi|,
\end{equation*}

which is directly proportional to the area S and the rate change with flux $\Phi$ of the transition frequency $\omega_{01}$. For our device we obtain $d\omega_{01}/d\Phi = -2\pi \times 5.3 GHz/\Phi_0$ at the bias point, resulting in $\mu = 1.10 \times 10^5 \mu_B$. 

Operating away from the bias point leads to a reduction of the decoherence time $T_2$ : ${T_2}^{-1} = (2T_1)^{-1} + {T_\Phi}^{-1}$, where $2T_1$ is a sum of relaxation time and $T_\Phi$ is a dephasing time. The dephasing rate appreciably increases at our bias point, which reduces $T_2$ and thus the number of available steps that can be implemented in the Kitaev algorithm. 

In the experiment we apply a Ramsey sequence of two consecutive $\pi/2$ pulses separated by a time delay $\tau$, which corresponds to an effective spin-1/2 precession around the z-axis of the Bloch sphere. The precession angle $\phi = \Delta \omega (\Phi) \tau$ is defined by the frequency mismatch $\Delta \omega(\Phi) = \omega_d - \omega_{01}(\Phi)$ between the transition frequency $\omega_{01}(\Phi)$ of the transmon qubit and the fixed drive frequency $\omega_d$ of the $\pi/2$ pulses. 








































\end{document}